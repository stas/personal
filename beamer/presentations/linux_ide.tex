\documentclass[compress]{beamer}

\usepackage{listings}
\usepackage{hyperref}

%\usetheme{OHS}
\usetheme[dark,framenumber]{OHS}
%\usetheme[darktitle,framenumber]{OHS}

\title{Programare sub GNU/Linux}
\subtitle{C\^{a}t de simplu poate fi?!}

\author{Tomi\c{t}\u{a} Militaru (\href{mailto:tomita.militaru@sblug.ro}{tomita.militaru@sblug.ro})
\\* \c{s}i
\\* Stas Su\c{s}cov (\href{mailto:stas@sblug.ro}{stas@sblug.ro})}
\date{Octomber 15th, 2009}

\begin{document}

% ----------------------------------------------------------------------------
% *** Titlepage <<<
% ----------------------------------------------------------------------------
\maketitle
% ----------------------------------------------------------------------------
% *** END of Titlepage >>>
% ----------------------------------------------------------------------------


\section{Introducere}
\subsection{Istoric}


% ----------------------------------------------------------------------------
% *** Frame <<<
% ----------------------------------------------------------------------------
\begin{frame}
\frametitle{Ce \^{i}nseamn\u{a} Cod Liber?}
\end{frame}
% ----------------------------------------------------------------------------
% *** END of Frame >>>
% ----------------------------------------------------------------------------


% ----------------------------------------------------------------------------
% *** Frame <<<
% ----------------------------------------------------------------------------
\begin{frame}
\frametitle{Ce \^{i}nseamn\u{a} Cod Liber?}

\begin{itemize}
    \item<1-> conceptul, de\c{s}i incomplet, a luat na\c{s}tere prin 1960
    \item<2-> reprezint\u{a} un mod de via\c{t}\u{a} \c{s}i de a crea
    \item<3-> a fost extins \c{s}i conturat de pasiona\c{t}ii de calculatoare
    \item<4-> a motivat crearea “The GNU Project”, “Free Software Foundation” (FSF) \c{s}i a “GNU General Public License” (GPL)
    \item<5-> adev\u{a}ratul gabarit a mi\c{s}c\u{a}rii s-a conturat prin 1990, odat\u{a} cu intrarea “Linux”-ului \^{i}n joc
    \item<6-> momentan reprezint\u{a} una dintre cele mai puternice mi\c{s}c\u{a}ri sociale: “Open Source Initiative” (OSI)
\end{itemize}

\end{frame}
% ----------------------------------------------------------------------------
% *** END of Frame >>>
% ----------------------------------------------------------------------------

\subsection{Rezultate}

% ----------------------------------------------------------------------------
% *** Frame <<<
% ----------------------------------------------------------------------------
\begin{frame}
\frametitle{Solu\c{t}ii libere de programare}

\begin{exampleblock}{GCC}
Colec\c{t}ia de compilatoare GNU.
\end{exampleblock}

\pause

\begin{exampleblock}{GTK+}
Sistem de dezvoltare a interfe\c{t}elor grafice (GNOME).
\end{exampleblock}

\pause

\begin{exampleblock}{Qt}
Sistem de dezvoltare a interfe\c{t}elor grafice (KDE).
\end{exampleblock}

\end{frame}
% ----------------------------------------------------------------------------
% *** END of Frame >>>
% ----------------------------------------------------------------------------


% ----------------------------------------------------------------------------
% *** Frame <<<
% ----------------------------------------------------------------------------
\begin{frame}
\frametitle{Solu\c{t}ii libere de programare}

\begin{exampleblock}{PHP}
Limbaj de programare interpretat orientat pentru aplica\c{t}ii web.
\end{exampleblock}

\pause

\begin{exampleblock}{Python}
Limbaj de programare interpretat orientat pentru aplica\c{t}ii de birou.
\end{exampleblock}

\pause

\begin{exampleblock}{Java}
Limbaj de programare obiectual foarte popular pentru aplica\c{t}ii de birou.
\end{exampleblock}

\pause

\begin{exampleblock}{SQLite}
Librarie portabil\u{a} \c{s}i mic\u{a}, ce ofer\u{a} suport pentru baze de date.
\end{exampleblock}

\end{frame}
% ----------------------------------------------------------------------------
% *** END of Frame >>>
% ----------------------------------------------------------------------------


% ----------------------------------------------------------------------------
% *** Frame <<<
% ----------------------------------------------------------------------------
\begin{frame}
\frametitle{Solu\c{t}ii libere de programare}

\begin{exampleblock}{MinGW}
Colec\c{t}ia de compilatoare GNU pentru Windows$^{\textregistered}$.
\end{exampleblock}

\pause

\begin{exampleblock}{Mono}
Alternativa liber\u{a} la proiectul .NET$^{\textregistered}$ \c{s}i C\#$^{\textregistered}$ de la Microsoft$^{\textregistered}$.
\end{exampleblock}

\pause

\begin{exampleblock}{SVN}
Sistem popular folosit la versionarea codului surs\u{a}.
\end{exampleblock}

\end{frame}
% ----------------------------------------------------------------------------
% *** END of Frame >>>
% ----------------------------------------------------------------------------

\section{Probleme \c{s}i solu\c{t}ii}
\subsection{Inpactul}

% ----------------------------------------------------------------------------
% *** Frame <<<
% ----------------------------------------------------------------------------
\begin{frame}
\frametitle{\^{I}nsemn\u{a}tatea \^{i}n mediul educa\c{t}ional}

\begin{itemize}
    \item<1-> Ofer\u{a} posibilitatea de a \^{i}nv\u{a}\c{t}a pe baza experien\c{t}ei altora.
    \item<2-> Ofer\u{a} libertatea de a \^{i}ncerca diferite limbaje de programare f\u{a}r\u{a} b\u{a}t\u{a}i de cap la configurare \c{s}i instalare.
    \item<3-> Asigur\u{a} posibilitatea de a crea o carier\u{a} prin aprofundarea cuno\c{s}tin\c{t}elor despre un limbaj de programare.
\end{itemize}

\end{frame}
% ----------------------------------------------------------------------------
% *** END of Frame >>>
% ----------------------------------------------------------------------------

\subsection{Exemple}

% ----------------------------------------------------------------------------
% *** Frame <<<
% ----------------------------------------------------------------------------
\begin{frame}
\frametitle{Exemplu de cod \^{i}n C}

\begin{alertblock}{Mediul de programare Code::Blocks}
\lstinputlisting[language=C,basicstyle=\ttfamily\footnotesize]{exemplu.c}
\end{alertblock}

\end{frame}
% ----------------------------------------------------------------------------
% *** END of Frame >>>
% ----------------------------------------------------------------------------


% ----------------------------------------------------------------------------
% *** Frame <<<
% ----------------------------------------------------------------------------
\begin{frame}
\frametitle{Exemplu de cod \^{i}n Python}

\begin{alertblock}{Mediul de programare Geany}
\lstinputlisting[language=Python,basicstyle=\ttfamily\tiny]{exemplu.py}
\end{alertblock}

\end{frame} 
% ----------------------------------------------------------------------------
% *** END of Frame >>>
% ----------------------------------------------------------------------------


% ----------------------------------------------------------------------------
% *** Frame <<<
% ----------------------------------------------------------------------------
\begin{frame}
\frametitle{Exemplu de cod \^{i}n HTML}

\begin{alertblock}{Mediul de programare BlueFish}
\lstinputlisting[language=HTML,basicstyle=\ttfamily\tiny]{exemplu.html}
\end{alertblock}

\end{frame} 
% ----------------------------------------------------------------------------
% *** END of Frame >>>
% ----------------------------------------------------------------------------


% ----------------------------------------------------------------------------
% *** Frame <<<
% ----------------------------------------------------------------------------
\begin{frame}
\frametitle{Exemplu de cod \^{i}n Java}

\begin{alertblock}{Mediul de programare Eclipse}
\lstinputlisting[language=JAVA,basicstyle=\ttfamily\tiny]{exemplu.java}
\end{alertblock}

\end{frame} 
% ----------------------------------------------------------------------------
% *** END of Frame >>>
% ----------------------------------------------------------------------------

\section{\^{I}ncheiere}
\subsection{\^{I}ntreb\u{a}ri}

% ----------------------------------------------------------------------------
% *** Frame <<<
% ----------------------------------------------------------------------------
\begin{frame}
\frametitle{\^{I}ntreb\u{a}ri?!?}

\end{frame} 
% ----------------------------------------------------------------------------
% *** END of Frame >>>
% ----------------------------------------------------------------------------

\subsection{Contacte}

% ----------------------------------------------------------------------------
% *** Frame <<<
% ----------------------------------------------------------------------------
\begin{frame}
\frametitle{Contacte}

\begin{itemize}
    \item Pagina grupului de utilizatori Linux din Sibiu \url{http://sblug.ro/}
    \item Pagina proiectului Open High School \url{http://school.sblug.ro/}
    \item Adres\u{a} email pentru contact \\* \href{mailto:salut@sblug.ro}{salut@sblug.ro}
\end{itemize}

\end{frame} 
% ----------------------------------------------------------------------------
% *** END of Frame >>>
% ----------------------------------------------------------------------------


\end{document}
